\title{\textbf{\Huge Matrix Batch Processing using CUDA}}

\author{{\Large Nguyen Duy Hoang, Harry (A0048242L)} \\
	{\Large Dandekar, Ashish (A0123873A)}\\
	{\Large Ku\"endig, Adrian (A0134824H)} \\
	{\Large Ranade, Ketki (A0119950B)}\\
	{\Large Pham Thanh Tung, Terry (A0120113J)} \\\\
	National University of Singapore,\\
	Singapore
}
\date{\today}

\documentclass[11pt]{article}
\usepackage[margin=1in]{geometry}
\usepackage[english]{babel}
% \usepackage{listingstyle}
\usepackage{graphicx}
\usepackage{caption}
\usepackage{subcaption}
% \usepackage{coz}
\usepackage{url}
\usepackage{array}
\usepackage{multirow}
% \usepackage[autolanguage]{numprint}

\begin{document}
\maketitle
\tableofcontents

\newpage
\section{Introduction}
Many modern applications such as multi-sensor networks require batch processing of large array of small-scale matrices. Leveraging on the massive multi-core architecture, our projects aim to explore the use of GPU to perform various matrix operations effectively and simultaneously. Unlike available libraries, the emphasis is not only on the efficient implementation of the primitives but also on batching the multiple matrices on GPUs. It embarks on first steps of numerous optimization problems like Gaussian Process Model where we conjecture that the massive parallelism of GPU will have cutting edge over CPU.
\subsection{Matrix Computation and CUDA}
\subsection{Gaussian Process Optimization}

\section{Related Work}
% Write about papers and existing libraries

Addition, multiplication and inversion are the primitive operations on the matrix, out of which inversion is far more non-trivial than other operations. Different papers have proposed solution to this problem. For example, Sharma et al. (2013) have implemented a fast parallel Gauss Jordan algorithm for matrix inversion on GPU. Benner et al. (2011) have shown that using the GPU only provides a speedup if the matrix exceeds a certain dimension. \\\\
The Gauss-Jordan algorithm discussed by Sharma et al. (2013) works with any invertible matrix. But the implementation is constrained by the limitations of CUDA (e.g., maximum of threads that can be launched and the size of the shared memory).\\\\
On the other hand, Benner et al. (2011) have focused on the SPD matrix inversion problem. The paper talks about two different ways to invert the SPD matrices namely, via Cholesky decomposition and using Gauss Jordan transformation. There is one remarkable result one gets from the paper that inverting the large matrices on GPU is profitable compared to inverting small matrices since the massive parallelism pays off the cost of data transfer.\\\\
Along with addition and multiplication, we aim to implement inversion of the matrices catered to the requirements enlisted above. Although Benner et al. (2011) have shown that inverting a small matrix on the GPU is not economical, they have not explored the possible performance gains, if any, achievable by batching inversion of multiple matrices. We will study these works and implement matrix operations by utilizing primitives provided by CUDA.


\section{Problem Description}

% Do mention the assumptions related to data (for instance: SPD, same matrix size within a batch)

\section{Algorithms}
% No need to write about addition/Multiplication
% If there is any need, do add the subsections
\subsection{Gauss-Jordan Method of Inversion}
Gauss-Jordan method is the traditional method to invert any kind of matrix. The method makes use of elementary row operations namely multiplication by scalar, linear combination of rows and row shuffling to invert the matrix. In this method, the original matrix($\mathcal{A}$) is augmented with identity matrix($\mathcal{I}$) of the same dimension. Row operations are performed so as to convert $\mathcal{A}$ to $\mathcal{I}$. The same set of operations are simultaneously applied to $\mathcal{I}$ which reduced it to $\mathcal{A}^{-1}$. \\\\
Gauss-Jordan method is an algorithmic way of finding inverse of the matrix since there is deterministic sequence of operations which yields the inverse of matrix. Let $A_i$ denotes the $i^{th}$ row of the matrix $\mathcal{A}$ and $a_{ij}$. Following is the set of row operations performed on each row $A_i$ sequentially:
\begin{description}
	\item[Pivoting] For a given row $\mathcal{A}_i$, the element on the principal diagonal of the matrix serves as the pivot element. If $a_{ii}$ is not zero then the control proceeds to the next step. If the $a_{ii}$ is zero then we need to shuffle the rows so as to make $a_{ii}$ non-zero. To do this, we search for the row $j$($> i$) such that $a_{ji}$ is not zero. If such a $j$ is found then $\mathcal{A}_{i}$ is swapped with $\mathcal{j}$. Failure at finding such a $j$ value implies that the matrix $\mathcal{A}$ is not invertible.
	\item[Normalization] Once a non-zero pivot is found, we divide $A_i$ by $a_{ii}$ to make the diagonal entry $1$.
	\item[Matrix Transformation] After the row $A_i$ has been normalized, we want to make elements $a_{ji}$, for all $j$ but for $i$, zero. This is achieved by transforming $j^{th}$ row as $A_j \leftarrow A_j - a_{ji}A_i$.
\end{description}
Same set operations are applied to $\mathcal{I}$ with the same scaars. When above procedure is applied to $\mathcal{A}$ for every row, the transformed $\mathcal{I}$ reflects the $\mathcal{A}^{-1}$.\\\\
Let's look at some caveats in the above procedure which one can exploit to parallelize the method. First of all, as stated earlier, these set of operations have to be performed sequentially on each row. This is the bottleneck in the parallelism which can not be alleviated. All of these operations when performed for a single row entirely changes the matrix and the scalars for the operations on the next row depend on the earlier transition. So, this part of the procedure has to be kept sequential. Despite this, we can indeed parallelize each operation. Let's see each operation one by one:
\begin{description}
	\item[Pivoting] If the element on the principal diagonal is already non-zero then there is no need to find pivot. Search for a non-zero element as pivot has to be sequential again so as to avoid race condiotion to return result. Once such an element is found, corresponding rows can be swapped parallely.
	\item[Normalization] Scalar-vector multiplication can, obviously, be done in parallel since we scale every element of the row by the same scalar.
	\item[Matrix Transformation] This is a non-trivial parallelization step. In this step, each row must be transformed. So a thread is spawned for each row to handle transformation of the corresponding row.
\end{description}

\subsection{Matrix Inversion using Cholesky Decomposition}

\section{Experimental Evaluation}

In this section we study the performance of the different inversion algorithms. For the experiments we compared the different inversion algorithms with a baseline algorithm implemented in C.

\subsection{System Specification}

All experiments were conducted on a CentOS 6.6 server machine with XXX hyperthreads XXX GHz each XXX kB L3/L2/L1 cache, XXX GB of memory and a NVIDIA XXX graphics card. The GPU contained XXX blocks, XXX warps, XXX threads per block, XXX kB of shared memory and 6 GB of global memory. The measured memory bandwidth between main memory and GPU memory was a bit more than 3 GB/s. All calculations were executed with single precision floating point math.

\subsection{Data Generation}

We used a synthetic dataset throughout our experiments. The dataset was generated using MATLAB and consists of multiple sets of square matrices. Each set contains exactly 100 matrices. To benchmark higher numbers of matrices we duplicated the matrices in the set until we got the required number of matrices. The sets of square matrices had dimension of the power of two starting from $8 \times 8$ until $256 \times 256$.

\subsection{Baseline Algorithms}

The baseline algorithm is implemented in C using the LAPACK library and it consists of only two calls to the library functions {\it spotrf} and {\it spotri}. We chose this algorithm as the sequential (in the sense of thread parallelism) baseline algorithm because it is straightforward to implement and efficient. We also conducted experiments with rudimentary parallelization using OpenMP to see the performance achievable on today's multicore processors. In this parallel algorithm we inverted multiple matrices in parallel using the OpenMP \textit{parallel for} statement.

In addition to the two algorithms implemented in this study we also implemented an inversion algorithm using the two cuda library functions {\it getrfBatched} and {\it getriBatched}.

\subsection{Benchmarking}

Our benchmarks consisted of 10 replicated calculations on the generated dataset. We increased the number of matrices starting from 100 in steps of 100 until 1’000. We included the memory transfer time to and from the device into the runtime of each algorithm. Naturally the baseline algorithm did not contain any memory transfer. Still, we argue that memory transfer plays a major role in the decision to offload computation to the GPU since it is a non-negligible time overhead.

In addition to the total runtime of each algorithm, we also report the total working memory required to calculate the inverse for each method. Finally, we also evaluated the numerical stability of the different algorithms. We did this by computing inverses using MATLAB and storing them along the initial matrices. The error we report is the sum of the absolute pairwise distances between the reference and the computed inverse matrix. We also thought about calculating the measuring the distance of the identity matrix and the result of multiplying the inverse matrix with the initial matrix. We discarded this idea because it would also include the error of the final matrix-matrix multiplication.

As a second evaluation method we implemented a mean and variance calculation as used in Gaussian Process Optimization (GPO). We again implemented the baseline algorithm in C as a sequential and as a parallel version using OpenMP. We expected the performance of the GPU to be better in this experiment than in the raw inversion experiment, since there are more operations involved in the mean and variance computation. As with the inversion benchmarks we report the runtime of each algorithm, the used working memory, and the absolute error to the solution obtained from MATLAB.

\section{Conclusion and Future Work}

\newpage

\bibliographystyle{abbrv}
\bibliography{references}
\end{document}
